\section{Installation et exécution du solveur}

\subsection{Installation}
    \paragraph{}
    Tout d'abord, le projet est disponible sur GitHub à l'adresse suivante :  \url{https://github.com/Asteip/KaboSolve}. Pour installer le solveur, exécuter les commandes suivantes dans un terminal :
    
    \begin{lstlisting}[language=bash,captionpos=b]
    # cloner le depot
    git clone https://github.com/Asteip/KaboSolve
    
    # installer le solveur
    cd KaboSolve
    make
    \end{lstlisting}
    
\subsection{Exécution}
    \paragraph{}
    Une fois le solveur installé, exécuter la commande suivante pour lancer le solveur :
    
    \begin{lstlisting}[language=bash,captionpos=b]
    
    ./KaboSolve <problem> <number of solutions> [<options>]

    \end{lstlisting}
    
    \paragraph{}
    L'argument <problem> prend les valeurs suivantes : 
    
    \begin{itemize}
        \item n-queens : problème des N-Queens, il faut ajouter l'option "number of queens" correspondant au nombre de reines (N). Exemple d'utilisation : \textit{./KaboSolve n-queens all 10}.
        
        \item more-money : problème du Send More Money. Exemple d'utilisation : \textit{./KaboSolve more-money all}.
        
        \item magic-square : problème du carré magique, il faut ajouter l'option "size of square" correspondant à la taille du carré. Exemple d'utilisation : \textit{./KaboSolve magic-square all 5}.
        
        \item sudoku / x-sodoku : problème du sudoku, il est possible de choisir entre "sudoku" qui ne prend pas en compte les diagonales de la grille et "x-sudoku" qui prend en compte les diagonales. Il faut ajouter l'option "size of the latin square" qui correspond à la taille N d'une "région" du sudoku, la taille de la grille sera donc de $N^{2} \times N^{2}$. Exemple d'utilisation : \textit{./KaboSolve sudoku all 3}.
    \end{itemize}
    
    \paragraph{}
    L'argument <number of solutions> prend les valeurs suivantes :
    
    \begin{itemize}
        \item one : exécution du solveur jusqu'à ce qu'il trouve une solution du problème.
        \item all : exécution du solveur jusqu'à ce qu'il trouve toutes les solutions du problème.
    \end{itemize}